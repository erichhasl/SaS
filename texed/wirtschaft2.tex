\documentclass{sasbase}

\usepackage[ngerman]{babel}
\usepackage{booktabs}

\begin{document}

\onecolumn
\title{Wirtschaft - Thesenpapier II}
\place{Ludwigsburg}
\datum{2. Juli 2018}
\edition{1}

\mytitle

\setlength{\parindent}{0mm}
\setlength{\parskip}{2mm}

\section{Lohnauszahlung}

Lohnauszahlungen finden morgens statt. Alle Mitarbeiter eines Betriebs sammeln sich also morgens
und erhalten ihren Lohn für den vorigen Tag.

Daraus ergibt sich folgender Tagesablauf:

\begin{enumerate}
    \item Lohnauszahlung
    \item Erste Schicht, Hälfte der Angestellten arbeitet, andere Hälfte konsumiert
    \item Zweite Schicht
    \item Tagesabrechnung beim Wirtschaftsministerium
\end{enumerate}

Am ersten Tag fällt die Lohnauszahlung durch die Betriebe weg, da jedem Bürger 60 G-Mark vom Staat
ausgezahlt werden (finanziert durch die von allen Bürgern eingezahlten 10€).

\section{Tagesabrechnung}

Am Ende jedes Tages müssen die Betriebe eine Tagesabrechnung durchführen, in der sie den getätigten
Umsatz, alle Lohnkosten und den Gewinn eintragen und dann im Wirtschaftsministerium vortragen.
Daraufhin wird die genaue Umsatzsteuer berechnet und direkt von den Betrieben bezahlt.

\section{Gewinnausschüttung}

Um eine bessere Verteilung der Gewinne zu gewährleisten, wird eine Mindest-Gewinnausschüttung von
beispielsweise 50\% an die Mitarbeiter empfohlen, die am nächsten Tag als Zusatzlohn ausgezahlt
wird. Der Gewinn berechnet sich aus dem erwirtschafteten Umsatz abzüglich Lohnkosten (für
Arbeitgeber wird der Durchschnittslohn der Mitarbeiter angenommen) und Ausgaben.

Im gleichen Zuge könnte der Mindestlohn dafür auf 1€ pro Stunde gesenkt werden.

\end{document}
