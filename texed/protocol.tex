\documentclass{sasbase}

\usepackage{lipsum}
\usepackage{enumitem}
\usepackage{graphicx}
\usepackage[ngerman]{babel}
\usepackage{tikz}
\usepackage{geometry}

\newcommand{\kreis}{
    $\vcenter{\begin{tikzpicture}
        \draw (0,0) circle (0.2cm);
    \end{tikzpicture}}$
}
\setlength{\parindent}{0mm}
\setlength{\parskip}{2mm}

\begin{document}

\title{Protokoll}

\place{Ludwigsburg}
\datum{19. Februar 2018}
\edition{1}
\onecolumn
\mytitle
\parensstyle

\section{Vor der Sitzung}

\begin{tabular}{p{4cm}p{14cm}}
    Fehlende Abgeordnete: & \dotfill \\
\end{tabular}

\begin{tabular}{p{4cm}p{14cm}}
    G\"{a}ste: &
    \begin{tabular}[t]{p{0.5cm}p{5cm}p{1cm}p{0.5cm}p{5cm}}
        \kreis & Pr\"{a}sident & & \kreis & Polizei-Chef \\[2mm]
        \multicolumn{5}{l}{Sonstige: \hspace{5mm} \dotfill} \\
    \end{tabular}
    \\
\end{tabular}

\begin{tabular}{p{4cm}p{14cm}}
    Leitung durch: &
    \begin{tabular}[t]{p{0.5cm}p{5cm}p{1cm}p{0.5cm}p{5cm}}
        \kreis & Pr\"{a}sident & & \kreis & Parlamentspr\"{a}sident \\[2mm]
        \multicolumn{5}{l}{Vertretung: \hspace{5mm} \dotfill} \\
    \end{tabular}
    \\
\end{tabular}

\vspace{3mm}
\begin{tabular}{p{4cm}p{14cm}}
    Beschlussf\"{a}higkeit: &
    \begin{tabular}[t]{p{0.5cm}p{5cm}p{1cm}p{0.5cm}p{5cm}}
        \kreis & < 10 Abgeordnete & & \kreis & $\geq$ 10 Abgeordnete  \\
    \end{tabular}
    \\
\end{tabular}

\vspace{3mm}
\begin{tabular}{p{4cm}p{14cm}}
    \raggedright Anwesenheit juristischer Berater: &
    \begin{tabular}[t]{p{0.5cm}p{5cm}p{1cm}p{0.5cm}p{5cm}}
        \kreis & Ja & & \kreis & Nein  \\
    \end{tabular}
    \\
\end{tabular}

\begin{tabular}{p{4cm}p{14cm}}
    Sitzungser\"{o}ffnung um: & \dotfill \\
\end{tabular}

\section{W\"{a}hrend der Sitzung}

\begin{tabular}{p{4cm}p{14cm}}
    Beschlüsse: & \dotfill \\[2mm]
    & \dotfill \\[2mm]
    & \dotfill \\[2mm]
    & \dotfill \\[2mm]
    & \dotfill \\[2mm]
    & \dotfill \\[2mm]
    & \dotfill \\[2mm]
    & \dotfill \\[2mm]
    & \dotfill \\[2mm]
    & \dotfill \\[2mm]
    & \dotfill \\[2mm]
    & \dotfill \\[2mm]
    & \dotfill \\[2mm]
    & \dotfill \\[2mm]
\end{tabular}

\begin{tabular}{p{4cm}p{14cm}}
    & \dotfill \\[2mm]
    & \dotfill \\[2mm]
    & \dotfill \\[2mm]
    & \dotfill \\[2mm]
    & \dotfill \\[2mm]
    & \dotfill \\[2mm]
\end{tabular}

\begin{tabular}{p{4cm}p{14cm}}
    Reden: & \dotfill \\[2mm]
    & \dotfill \\[2mm]
    & \dotfill \\[2mm]
    & \dotfill \\[2mm]
    & \dotfill \\[2mm]
    & \dotfill \\[2mm]
    & \dotfill \\[2mm]
    & \dotfill \\[2mm]
    & \dotfill \\[2mm]
    & \dotfill \\[2mm]
    & \dotfill \\[2mm]
    & \dotfill \\[2mm]
    & \dotfill \\[2mm]
    & \dotfill \\[2mm]
    & \dotfill \\[2mm]
    & \dotfill \\[2mm]
    & \dotfill \\[2mm]
\end{tabular}

\vspace{3mm}
\begin{tabular}{p{4cm}p{14cm}}
    R\"{u}gen: &
    \begin{tabular}[t]{p{0.5cm}p{5cm}p{0.2cm}p{1cm}p{5.3cm}}
        An & \dotfill & & Grund & \dotfill \\[2mm]
        An & \dotfill & & Grund & \dotfill \\[2mm]
        An & \dotfill & & Grund & \dotfill \\[2mm]
        An & \dotfill & & Grund & \dotfill \\[2mm]
    \end{tabular}
    \\
\end{tabular}

\section{Nach der Sitzung}

\begin{tabular}{p{4cm}p{14cm}}
    Unterschrift Leitung: & \dotfill \\
\end{tabular}
\vspace*{1mm}

\begin{tabular}{p{4cm}p{14cm}}
    Unterschrift Ger\"{u}gte: & \dotfill \\
\end{tabular}

\end{document}
