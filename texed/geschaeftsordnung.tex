\documentclass{sasbase}

\usepackage{lipsum}
\usepackage{enumitem}
\usepackage{graphicx}

\begin{document}

\title{Gesch\"{a}ftsordnung d. Parlaments}
\place{Ludwigsburg}
\datum{19. November 2017}
\edition{1}

\setcounter{secnumdepth}{5}

\mytitle

% OPTIONAL
%\squarestyle
% OR
\parensstyle

\section{Gesch\"{a}ftsordnung d. Parlaments}
\segmentoflaw{Pr\"{a}ambel}
Wenn im Folgenden die weibliche Form verwendet wurde, so ist die m\"{a}nnliche Form nat\"{u}rlich mit inbegriffen. Diese Vereinfachung diente allein der besseren Lesbarkeit. Diese Gesch\"{a}ftsordnung ist ein Entwurf, den das Parlament nach Konstitution ratifizieren und \"{u}berarbeiten wird. 
\segmentoflaw{Gesch\"{a}ftsordnung}
\begin{article}{Beschlussfähigkeit}
	\item Das Parlament ist beschlussfähig, wenn es ordnungsgemäß einberufen wurde und mindestens 10 Abgeordnete anwesend sind.
	\item Zu Beginn der Versammlung wird die Beschlussfähigkeit des Parlaments durch die Parlamentspr\"{a}sidentin festgestellt. Spätere Feststellungen der Beschlussfähigkeit bedürfen eines Antrags.
	\item Ist das Parlament nicht beschlussfähig, kann die Parlamentspr\"{a}sidentin eine weiteren Sitzung eine Woche sp\"{a}ter mit selber Tagesordnung einberufen. Dieses Parlament ist in jedem Fall beschlussfähig.
\end{article}
\begin{article}{Anträge}
	\item Allgemeine Antr\"{a}ge m\"{u}ssen von den Abgeordneten mindestens 1 Woche vor Sitzung eingereicht werden, Gesetzesvorschl\"{a}ge zus\"{a}tzlich von 5 Abgeordneten oder der Regierung.
	\item Über einen nicht fristgerecht eingereichten Antrag (Dringlichkeitsantrag) wird nur verhandelt, wenn er schriftlich bei der Parlamentspr\"{a}sidentin eingereicht wird und vom Parlament in einer Abstimmung als dringlich anerkannt wird. Anträge auf Änderung der Geschäftsordnung des Parlamentes können nicht als dringlich behandelt werden.
\end{article}
\begin{article}{Geschäftsordnungsantr\"{a}ge}
	\item Geschäftsordnungsanträge zur Regelung des Verfahrens des Parlaments können jederzeit gestellt werden. Sie sind umgehend zu behandeln und unterbrechen die Behandlung des laufenden Tagesordnungspunktes. Vor der Entscheidung über den Geschäftsordnungsantrag darf die Behandlung des laufenden Tagesordnungspunktes nicht fortgesetzt werden.
	\item Bei Geschäftsordnungsanträgen ist eine Rednerin  für und ein Rednerin gegen den Geschäftsordnungsantrag zu hören. Dann erfolgt sofort die Abstimmung über den Geschäftsordnungsantrag.
	\item Zulässige Geschäftsordnungsanträge sind beispielsweise:
		\begin{enumerate}
			\item Antrag auf Schluss der Debatte und sofortige Abstimmung   Antrag auf Schluss der Redeliste
			\item Antrag auf Begrenzung der Redezeit
			\item Antrag auf Vertagung
			\item Antrag auf Unterbrechung der Versammlung
			\item Antrag auf Feststellung der Beschlussfähigkeit   
			\item Antrag auf Verweisung an ein anderes Gremium
		\end{enumerate}
	\item Anträge auf Schluss der Debatte und sofortige Abstimmung, Schluss der Redeliste oder Begrenzung der Redezeit können nur von solchen stimmberechtigten Abgeordneten gestellt werden, die selbst zur Sache noch nicht gesprochen haben.
\end{article}
\begin{article}{Abstimmungen}
	\item Das Parlament beschließt grundsätzlich mit der einfachen Mehrheit der abgegebenen Stimmen. Stimmenthaltungen werden nicht mitgezählt. Bei Stimmengleichheit ist ein Antrag abgelehnt.
	\item Die Abstimmung erfolgt grundsätzlich offen, wenn nicht mindestens f\"{u}nf stimmberechtigte Abgeordnete des Parlaments eine schriftliche und geheime Abstimmung verlangen.
\end{article}
\begin{article}{Wahl}
\item Zur Durchführung von Wahlen beruft das Parlament einen Wahlausschuss von drei Abgeordneten. Hierbei \"{u}bernimmt die Parlamentspr\"{a}sidentin die Leitung, und jeweils ein Abgeordneten von Opposition und Regierung.
\item Die Parlamentspr\"{a}sidentin fordert die stimmberechtigten Abgeordneten auf, Kandidatinnen vorzuschlagen. Die Parlamentspr\"{a}sidentin befragt die Kandidaten und Kandidatinnen, ob sie kandidieren möchten.
\item Eine Abwesende kann gewählt werden, wenn dem Parlamentspr\"{a}sidium vor der Wahl eine schriftliche Erklärung vorliegt, dass der Abwesende bzw. die Abwesende bereit ist, zu kandidieren und im Fall der Wahl diese anzunehmen.
\item Wahlen erfolgen schriftlich und geheim, wenn das Parlament nicht einstimmig die offene Wahl beschließt.
\item Gewählt ist, wer die absolute Mehrheit der abgegebenen Stimmen auf sich vereinigen kann. Stimmenthaltungen werden nicht mitgezählt. Kommt eine absolute Mehrheit nicht zustande, findet ein zweiter Wahlgang statt, in dem nur noch die beiden Kandidatinnen zur Wahl stehen, die im ersten Wahlgang die meisten Stimmen erhalten haben.
\end{article}
\begin{article}{Protokoll}
\item Über jede Parlamentssitzung ist ein Beschlussprotokoll anzufertigen. Das Protokoll ist von der Pr\"{a}sidentin zu unterzeichnen.
\item Dieses Protokoll wird ver\"{o}ffentlicht.	
\end{article}
\section{Impressum}
\begin{minipage}{0.4\linewidth}
\includegraphics[width=\textwidth]{apb_icon.png}
\end{minipage}
\begin{minipage}{0.5\linewidth}
{\raggedright stellvertretend Christian Merten und Nils Hebach.}
\end{minipage}
\end{document}


