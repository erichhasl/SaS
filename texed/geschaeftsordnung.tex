\documentclass{sasbase}

\usepackage{lipsum}
\usepackage{enumitem}
\usepackage{graphicx}
\usepackage[ngerman]{babel}

\begin{document}

\title{Gesch\"{a}ftsordnung des Parlaments}
\place{Ludwigsburg}
\datum{19. November 2017}
\edition{1}

\setcounter{secnumdepth}{5}

\mytitle

% OPTIONAL
%\squarestyle
% OR
\parensstyle

\section{Gesch\"{a}ftsordnung des Parlaments}
\segmentoflaw{Pr\"{a}ambel}
Wenn im Folgenden die weibliche Form verwendet wurde, so ist die m\"{a}nnliche Form nat\"{u}rlich mit inbegriffen. Diese Vereinfachung diente allein der besseren Lesbarkeit. Diese Gesch\"{a}ftsordnung ist ein Entwurf, den das Parlament nach Konstitution ratifizieren und \"{u}berarbeiten wird. 
\segmentoflaw{Gesch\"{a}ftsordnung}
\begin{article}[Beschlussfähigkeit]
	\item Das Parlament ist beschlussfähig, wenn es ordnungsgemäß einberufen wurde und mindestens 10 Abgeordnete anwesend sind.
	\item Zu Beginn der Versammlung wird die Beschlussfähigkeit des Parlaments durch die Parlamentspr\"{a}sidentin festgestellt. Spätere Feststellungen der Beschlussfähigkeit bedürfen eines Antrags.
	\item Ist das Parlament nicht beschlussfähig, kann die Parlamentspr\"{a}sidentin eine weitere Sitzung eine Woche bzw. während der Staatslaufzeit einen Tag später mit der selben Tagesordnung einberufen. Dieses Parlament ist in jedem Fall beschlussfähig.
\end{article}
\begin{article}[Anträge]
	\item Allgemeine Antr\"{a}ge m\"{u}ssen von den Abgeordneten mindestens einen Tag vor Sitzung eingereicht werden, Gesetzesvorschl\"{a}ge zus\"{a}tzlich von 5 Abgeordneten oder der Regierung.
	\item Über einen nicht fristgerecht eingereichten Antrag (Dringlichkeitsantrag) wird nur verhandelt, wenn er schriftlich bei der Parlamentspr\"{a}sidentin eingereicht wird und vom Parlament in einer Abstimmung als dringlich anerkannt wird. Anträge auf Änderung der Geschäftsordnung des Parlamentes können nicht als dringlich behandelt werden.
\end{article}
\begin{article}[Geschäftsordnungsantr\"{a}ge]
	\item Geschäftsordnungsanträge zur Regelung des Verfahrens des Parlaments können jederzeit gestellt werden. Sie sind umgehend zu behandeln und unterbrechen die Behandlung des laufenden Tagesordnungspunktes. Vor der Entscheidung über den Geschäftsordnungsantrag darf die Behandlung des laufenden Tagesordnungspunktes nicht fortgesetzt werden.
	\item Bei Geschäftsordnungsanträgen ist eine Rednerin  für und eine Rednerin gegen den Geschäftsordnungsantrag zu hören. Dann erfolgt sofort die Abstimmung über den Geschäftsordnungsantrag.
	\item Zulässige Geschäftsordnungsanträge sind beispielsweise:
		\begin{enumerate}
			\item Antrag auf Schluss der Debatte und sofortige Abstimmung   Antrag auf Schluss der Redeliste
			\item Antrag auf Begrenzung der Redezeit
			\item Antrag auf Vertagung
			\item Antrag auf Unterbrechung der Versammlung
			\item Antrag auf Feststellung der Beschlussfähigkeit   
			\item Antrag auf Verweisung an ein anderes Gremium
		\end{enumerate}
	\item Anträge auf Schluss der Debatte und sofortige Abstimmung, Schluss der Redeliste oder Begrenzung der Redezeit können nur von solchen stimmberechtigten Abgeordneten gestellt werden, die selbst zur Sache noch nicht gesprochen haben.
\end{article}
\begin{article}[Abstimmungen]
	\item Das Parlament beschließt grundsätzlich mit der einfachen Mehrheit der abgegebenen Stimmen. Stimmenthaltungen werden nicht mitgezählt. Bei Stimmengleichheit ist ein Antrag abgelehnt.
	\item Die Abstimmung erfolgt grundsätzlich offen, wenn nicht mindestens f\"{u}nf stimmberechtigte Abgeordnete des Parlaments eine schriftliche und geheime Abstimmung verlangen.
\end{article}
\begin{article}[Wahl]
\item Zur Durchführung von Wahlen beruft das Parlament einen Wahlausschuss von drei Abgeordneten. Hierbei \"{u}bernimmt die Parlamentspr\"{a}sidentin die Leitung, und jeweils eine Abgeordnete von Opposition und Regierung.
\item Die Parlamentspr\"{a}sidentin fordert die stimmberechtigten Abgeordneten auf, Kandidatinnen vorzuschlagen. Die Parlamentspr\"{a}sidentin befragt die Kandidaten und Kandidatinnen, ob sie kandidieren möchten.
\item Eine Abwesende kann gewählt werden, wenn dem Parlamentspr\"{a}sidium vor der Wahl eine schriftliche Erklärung vorliegt, dass der Abwesende bzw. die Abwesende bereit ist, zu kandidieren und im Fall der Wahl diese anzunehmen.
\item Wahlen erfolgen schriftlich und geheim, wenn das Parlament nicht einstimmig die offene Wahl beschließt.
\item Gewählt ist, wer die absolute Mehrheit der abgegebenen Stimmen auf sich vereinigen kann. Stimmenthaltungen werden nicht mitgezählt. Kommt eine absolute Mehrheit nicht zustande, findet ein zweiter Wahlgang statt, in dem nur noch die beiden Kandidatinnen zur Wahl stehen, die im ersten Wahlgang die meisten Stimmen erhalten haben.
\item Bei mehrfacher Besetzung eines Amts stimmt jede Abgeordnete für die der Besetzungszahl des Amts
    entsprechende Anzahl an Kandidatinnen. Bei Stimmgleichheit erfolgt eine Stichwahl.
\end{article}
\begin{article}[Parlamentspräsidentin]
    \item Die Parlamentspräsidentin hält das Hausrecht für die Räumlichkeiten des Parlaments
        insbesondere während der Sitzungen
    \item Aufgaben der Parlamentspräsidentin
    \begin{enumerate}
        \item Festlegung der Tagesordnung gemäß Artikel 2
        \item Ordentliche Einladung der Abgeordneten
        \item Eröffnung der Sitzung
        \item Leitung der Sitzung
        \item Vergabe von Redezeiten
        \item Kontrolle der Einhaltung der Geschäftsordnung
        \item Schließung der Sitzung
    \end{enumerate}
    \item Bei Verstoß gegen die Geschäftsordnung oder bei Störung des Sitzungsablaufs kann die
        Parlamentspräsidentin Rügen gegen einzelne Abgeordnete ausprechen. Nach Ermessen der Parlamentspr\"{a}sidentin 
        wird die Abgeordnete aus der Sitzung entfernt. Bei spätestens der zweite Rüge
        innerhalb einer Sitzung muss die Abgeordnete den Saal verlassen, darf aber weiterhin abstimmen.
    \item Bis zur Wahl der Parlamentspräsidentin oder im Falle ihrer Abwesenheit wird diese durch
        die Präsidentin vertreten.
\end{article}

\begin{article}[Redezeiten]
    \item Die einer Fraktion maximal zustehende Redezeit ergiebt sich aus der Anzahl der Abgeordneten, die sie stellt.
    \item Diese Gesamtredezeit kann nun auf beliebig viele Abgeordnete aufgeteilt werden.
    \item Empfohlen ist eine Redezeit von 2 Minuten f\"{u}r jeden Abgeordneten, der Wert liegt allerdings im Ermessen der Parlamentspr\"{a}sidentin.
\end{article}
    
\begin{article}[Einladung]
    \item Die Einladung zur Parlamentssitzung erfolgt über mindestens einen der folgenden
        Informationswege:
        \begin{enumerate}
            \item Email an jede einzelne Abgeordnete
            \item Verkündigung am Vertretungsplan
            \item Aushang in der SMV Vitrine
        \end{enumerate}
    \item Die Einladung muss mindestens ein Tag vor der Sitzung unter Nennung des genauen
        Tagungszeitpunkts und unter Verweis auf die Tagesordnung erfolgen.
\end{article}
\begin{article}[Protokoll]
\item Über jede Parlamentssitzung ist ein Beschlussprotokoll anzufertigen. Das Protokoll ist von der Parlamentspräsidentin zu unterzeichnen.
\item Dieses Protokoll wird ver\"{o}ffentlicht.	
\end{article}
\section{Impressum}
\begin{minipage}{0.4\linewidth}
\includegraphics[width=\textwidth]{apb_icon.png}
\end{minipage}
\begin{minipage}{0.5\linewidth}
{\raggedright stellvertretend Christian Merten und Nils Hebach.}
\end{minipage}
\end{document}
