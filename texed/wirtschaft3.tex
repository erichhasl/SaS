\documentclass{sasbase}

\usepackage[ngerman]{babel}
\usepackage{booktabs}

\begin{document}

\onecolumn
\title{Wirtschaft - Thesenpapier III}
\place{Ludwigsburg}
\datum{11. Juli 2018}
\edition{1}

\mytitle

\setlength{\parindent}{0mm}
\setlength{\parskip}{2mm}

\section{Wirtschaftskontrolldienst}

Der WkD ist für die Kontrolle der Betriebe zuständig. Dies findet in Zweierteams statt. Teile
des WkD gehen verdeckt vor und beobachten unauffällig die Abläufe in den Betrieben und notiert
Auffälligkeiten. Andere kontrollieren sich als WkD ausweisend die Betriebe. Dabei soll
insbesondere auf folgende Dinge geachtet werden.

\begin{itemize}
    \item Korrekte Mülltrennung
    \item Ordnungsgemäße Aufbewahrung von Lebensmitteln
    \item Einhaltung der Nachhaltigkeitsverordnung (Verbot des Verkaufs von Plastikflaschen)
    \item Ordentliche Kassenführung
\end{itemize}

\section{Zoll}

Alle Zollbeamten der ersten Schicht sind morgens ab spätestens 7:30 Uhr im Staatsgebiet,
bauen die Zollstände auf und nehmen Ausweiskontrollen vor. Der genaue Aufbauplan wird von der
Wirtschaftsministerin am Vorbereitungstag verkündet. Für die Anwesenheitskontrolle liegen Listen
vor, in der die Identifikationsnummer und Spalten für die Anwesenheit für Dienstag bis Freitag
abgetragen sind.

Am Ende des Projekts werden die Anwesenheitslisten der entsprechenden Klassen den Klassenlehrern
abgegeben. Alle Schüler, die teilweise nicht anwesend waren, müssen sich wie gewohnt bis spätestens
Montag 23. Juli bei ihrem Klassenlehrer entschuldigen.

\end{document}
