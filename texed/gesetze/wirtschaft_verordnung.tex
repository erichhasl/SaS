\documentclass{sasbase}

\usepackage{lipsum}
\usepackage{enumitem}
\usepackage{graphicx}

\begin{document}

\title{Verordnung zur Wirtschaftsordnung}
\place{Ludwigsburg}
\datum{18. Juni 2018}
\edition{1}

\setcounter{secnumdepth}{5}

\mytitle

% OPTIONAL
%\squarestyle
% OR
\parensstyle

\section{Rahmenverordnungen}

\begin{lawparagraph}[Schichten]
	\item Täglich sind zwei Arbeitsschichten mit jeweils drei Stunden festgelegt.
    \item Arbeitnehmerinnen arbeiten täglich entweder in der ersten Schicht oder in der
        zweiten Schicht.
    \item Von Abs. 2 ausgenommen sind Wahlbeamtinnen, Funktionärinnen und Betriebsleiterinnen.
\end{lawparagraph}

\begin{lawparagraph}[Rahmenfinanzierung]
    \item Für die Finanzierung der Betriebe wird eine Gebühr von 10€ erhoben.
    \item Davon werden 4€ einbehalten und 6€ in G-Mark an die Bürgerinnen ausgezahlt.
    \item Am Ende des Projekts erhalten alle Bürgerinnen maximal wieder 10€.
\end{lawparagraph}

\section{Betriebe}

\begin{lawparagraph}[Betriebskredite]
    \item Betriebe müssen vor Beginn der Zeit einen Wirtschaftsplan vorlegen, der die genauen
        Einnahmen und Ausgaben vorrausplant.
    \item Nach Ermessen des Wirtschaftsministeriums werden den Betrieben unter Berücksichtigung
        des vorgelegten Wirtschaftsplans Kredite gewährleistet, die zum Ende der Zeit zurückgezahlt
        werden.
\end{lawparagraph}

\begin{lawparagraph}[Rücktausch von G-Mark in Euro]
    \item Der Rücktausch von G-Mark in Euro ist nicht möglich.
    \item Von Abs. 1 ist die Finanzierung neuer Importe der Betriebe ausgenommen.
\end{lawparagraph}

\begin{lawparagraph}[Umsatzsteuer]
    \item Auf Umsätze wird eine Umsatzsteuer von 25\% an den Staat abgeführt.
    \item Von Abs. 1 ist der Zwischenhandel ausgenommen.
    \item Für die korrekte Abführung der Steuer ist die Betriebsleitung zuständig.
    \item Zuwiderhandlungen gegen Abs. 1 und 3 regelt das Strafgesetzbuch.
\end{lawparagraph}

\begin{lawparagraph}[Währung]
    \item Die Staatswährung heißt G-Mark.
    \item Es besteht eine feste Kopplung der G-Mark an die Währung des Nachbarlands, Euro, von 10:1.
\end{lawparagraph}

\begin{lawparagraph}[Mindestlohn]
    \item Jede Bürgerin muss mit mindestens 15 G-Mark pro Stunde entlohnt werden.
    \item Für die Einhaltung des Mindestlohns ist die Betriebsleitung zuständig.
    \item Zuwiderhandlungen gegen Abs. 1 und 2 regelt das Strafgesetzbuch.
\end{lawparagraph}

\begin{lawparagraph}[Zoll und Visa]
    \item Ausländer müssen ein Visum beantragen.
    \item Dieses kostet für Erwachsene 10€, wobei 8€ gegen G-Mark getauscht und 2€ als Gebühr
        einbehalten werden.
    \item Kinder unter 14 Jahren zahlen 6€, wobei 5€ gegen G-Mark getauscht und 1€ als Gebühr
        einbehalten werden.
    \item Familien müssen nur die ersten zwei Kinder bezahlen.
    \item Für die korrekte Abführung der Visagebühren sind die Zollbeamtinnen verantwortlich.
\end{lawparagraph}

\end{document}
