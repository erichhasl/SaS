\documentclass{sasbase}

\usepackage{lipsum}
\usepackage{enumitem}
\usepackage{graphicx}

\begin{document}

\title{Strafgesetzbuch}
\place{Ludwigsburg}
\datum{18. Juni 2018}
\edition{1}

\setcounter{secnumdepth}{5}

\mytitle

% OPTIONAL
%\squarestyle
% OR
\parensstyle

\section{Rahmenverordnungen}

\begin{lawparagraph}[Existenzminimum]
	\item Durch die Verhängung einer Geldstrafe darf das Vermögen einer Bürgerin nicht unter
        5 G-Mark fallen.
    \item Von Abs. 1 können in schwerwiegenden Fällen Ausnahmen gemacht werden.
\end{lawparagraph}

\begin{lawparagraph}[Sanktionsbestände]
    \item Ein Tagessatz ist das Einkommen, das in einer Schicht erzielt wird.
    \item Das Einkommen in Abs. 1 bemisst sich bei Betriebsleiterinnen am durchschnittlichen Gewinn
        des Betriebs.
\end{lawparagraph}

\section{Privatpersonen}

\begin{lawparagraph}[Diebstahl und Sachbeschädigung]
	\item Unerlaubtes in Besitzbringen, Entwenden oder Beschädigen von Gegenständen mit einem
        Sachwert unter 5 Euro und keinem größeren ideellen Wert.
    \item Sanktionierung durch 0,5 - 1 Tagessatz und Erstattung des Gegenstands samt enstandener
        Schäden
    \item Die in Abs. 1 ausgenommenen Bestände unterliegen dem Strafrecht des Nachbarlands.
\end{lawparagraph}

\begin{lawparagraph}[Diskriminierung]
    \item Vorsätzliches Benachteiligen oder Bevorzugen einer Person aufgrund der Zugehörigkeit zu einer
        Gruppe.
    \item 0,25-0,5 Tagessätze an die Geschädigte
\end{lawparagraph}

\begin{lawparagraph}[Nötigung]
    \item Zwingen zu einer Handlung, Duldung oder Unterlassung unter Androhung von Gewalt oder
        verwerflichen Konsequenzen.
    \item 0,25-1 Tagessätze an die Geschädigte
\end{lawparagraph}

\begin{lawparagraph}[Beleidigung]
    \item Kundgabe der Missachtung oder Nichtachtung durch Werturteile
    \item 0,25-0,5 Tagessätze an die Geschädigte
\end{lawparagraph}

\begin{lawparagraph}[Betrung und Hochstapelei]
    \item Erschleichen von Leistungen oder Dingen von Wert unter Vorspielung falscher Tatsachen.
    \item 0-1 Tagessatz an die Geschädigte
\end{lawparagraph}

\section{Betriebe}

\begin{lawparagraph}[Kartellbildung]
    \item Vertrag oder Beschluss zwischen selbstständig bleibenden Unternehmern oder sonstigen
        Marktakteuren der gleichen Marktseite zur Beschränkung ihres Wettbewerbes.
    \item Maximal ein Tagesumsatz
\end{lawparagraph}

\begin{lawparagraph}[Lohndumping]
    \item Unterwandern der Mindestlohngrenze oder verwerfliche und menschenunwürdige
        Gehaltszahlungen.
    \item Nachzahlung des ausstehenden Lohns an die Geschädigte und maximal eine Geldstrafe in Höhe
        eines Tagesumsatzes.
\end{lawparagraph}

\begin{lawparagraph}[Diskriminierung]
    \item Vorsätzliches Benachteiligen oder Bevorzugen einer Person aufgrund der Zugehörigkeit
        zu einer Gruppe.
    \item Vollen Mindestlohnzahlung an die Geschädigte bis zum Ende der Tage.
\end{lawparagraph}

\begin{lawparagraph}[Steuerhinterziehung]
    \item Vorsätzliches Nichtzahlen von Steuern in voller Höhe oder Unterschlagen von Tatsachen.
    \item Rückzahlung des Steuerbetrags und maximal eine Geldstrafe in Höhe eines Tagesumsatzes.
\end{lawparagraph}

\end{document}
