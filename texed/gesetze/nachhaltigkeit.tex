\documentclass{sasbase}

\usepackage{lipsum}
\usepackage{enumitem}
\usepackage{graphicx}

\begin{document}

\title{Verordnung zur Nachhaltigkeit}
\place{Ludwigsburg}
\datum{18. Juni 2018}
\edition{1}

\setcounter{secnumdepth}{5}

\mytitle

% OPTIONAL
%\squarestyle
% OR
\parensstyle

\section{Müllregelung}

\begin{lawparagraph}[Müllkontrolldienst]
	\item Der Wirtschaftskontrolldienst ist für die Kontrolle der korrekten Mülltrennung
        nach dem im umliegenden Nachbarland geregelten Müllsystem und die Kontrolle einer
        angemessenen Müllmenge verantwortlich.
    \item Das Innenministerium ist verantwortlich für die Koordinierung des Müllkontrolldiensts.
\end{lawparagraph}

\begin{lawparagraph}[Müllsteuer]
    \item Auf übermäßige Mengen Müll wird keine Steuer erhoben.
    \item Das Innenministerium meldet übermäßige Müllmengen an das Parlament und prüft die
        Einführung einer Müllsteuer regelmäßig.
\end{lawparagraph}

\section{Plastikflaschen}

\begin{lawparagraph}[Glasflaschenpflicht]
    \item Betriebe dürfen Getränke nur in Glasflaschen verkaufen.
    \item Von §2 Abs. 1 ist die Abfüllung in selbst mitgebrachte Behältnisse ausgenommen.
\end{lawparagraph}

\begin{lawparagraph}[Freies Wasser]
    \item Jeder Bürgerin muss freier Zugang zu Trinkwasser, abfüllbar in selbst mitgebrachte
        Behältnisse, gewährt werden.

\end{document}
