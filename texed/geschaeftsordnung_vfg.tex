\documentclass{sasbase}

\usepackage{lipsum}
\usepackage{enumitem}
\usepackage{graphicx}
\usepackage[ngerman]{babel}

\begin{document}

\title{Gesch\"{a}ftsordnung der Gerichte}
\place{Ludwigsburg}
\datum{13. März 2018}
\edition{1}

\setcounter{secnumdepth}{5}

\mytitle

% OPTIONAL
%\squarestyle
% OR
\parensstyle

\section{Gesch\"{a}ftsordnung der Gerichte}
\segmentoflaw{Pr\"{a}ambel}
Wenn im Folgenden die weibliche Form verwendet wurde, so ist die m\"{a}nnliche Form nat\"{u}rlich mit inbegriffen. Diese Vereinfachung dient allein der besseren Lesbarkeit.

\segmentoflaw{Gesch\"{a}ftsordnung}

\begin{article}[Beschlussfähigkeit]
	\item Das Gericht ist beschlussfähig, wenn im Falle des Verfassungsgerichts mindestens zwei,
        im Falle des Zivil- und Strafgerichts mindestens 3 Richterinnen anwesend sind und
        ordnungsgemäß eingeladen wurde.
	\item Zu Beginn der Verhandlung wird die Beschlussfähigkeit des Gerichts durch die Vorsitzende
        des Gerichts festgestellt.
\end{article}

\begin{article}[Vorsitz]
    \item Die Richterinnen wählen zu Beginn ihrer Amtszeit eine Vorsitzende.
    \item Aufgaben der Vorsitzenden sind
        \begin{enumerate}
            \item Leitung der Verhandlungen
            \item Einladung der Richterinnen und Verhandlungsteilnehmerinnen
            \item Koordination der Terminfindung
            \item Festlegung der Verhandlungsgegenstände
        \end{enumerate}
    \item Gegen Vorschlag einer neuen Vorsitzenden kann dieses Amt jeder Zeit neu gewählt werden.
\end{article}

\begin{article}[Einladung]
    \item Zu einer Verhandlung muss den Verhandlungsgegenständen entsprechend früh eingeladen
        werden.
    \item Dabei soll besondere Sorgfalt bei der Einladung nicht-ständiger Verhandlungsteilnehmerinnen angewendet werden.
    \item Einladung erfolgt über zeitgemäße Kommunikationswege.
    \item Die Vorsitzende ist für die ordnungsgemäße Einladung verantwortlich.
\end{article}

\begin{article}[Verhandlung]
    \item Die Verhandlung findet öffentlich statt.
    \item Zu Zwecken der Urteilsberatung können alle nicht-ständigen Verhandlungsteilnehmerinnen
        temporär von der Verhandlung ausgeschlossen werden.
    \item Zur Aufrechterhaltung der Verhandlungsordnung kann die Vorsitzende Anwesende der
        Verhandlung verweisen.
    \item Ein Protokoll mit Urteilen und Begründungen wird angefertigt und veröffentlicht.
\end{article}
\end{document}
