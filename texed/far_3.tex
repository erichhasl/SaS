\documentclass{sasbase}

\usepackage{lipsum}
\usepackage{enumitem}
\usepackage{graphicx}

\begin{document}

\title{Informationsblatt der APB}
\place{Ludwigsburg}
\datum{05. Februar 2018}
\edition{3}

\setcounter{secnumdepth}{5}

\mytitle

% OPTIONAL
%\squarestyle
% OR
\parensstyle

\section{Ankündigungen}
\begin{itemize}
    \item Am 09.02. schließen die Bewerbungen von Richterinnen, Staatssekretärinnen und Staatsanwältinnen.
    \item In der Woche vom 19.02. bis 23.02. tagt das Parlament zum ersten Mal.
    \item Ab dem 01.04. werden Betriebsgründungen akzeptiert.
\end{itemize}

\section{Wahlergebnis}

\section{Interviews}

\topic{Henrik Möller, Vorsitzender der LSD}

\begin{question}{Erst einmal herzlichen Glückwunsch zum Wahlerfolg. Die LSD zieht mit Abstand als stärkste Kraft in das Parlament - hat  Sie das überrascht?}
    Wir sind sehr erfreut und auch ein wenig überrascht über den doch sehr deutlichen Sieg. 
    Ich möchte die Gelegenheit nutzen, um  mich  bei allen Mitgliedern und Unterstützern zu bedanken, ohne die dieser Erfolg nicht möglich gewesen wäre. 
    Während dem Wahlkampf  zeigte unsere Partei, wie gut Unterstufe, Mittelstufe und Oberstufe in unserer Partei zusammenarbeiten können.
    Wir haben beispielsweise einen Kuchenverkauf organisiert und den Gewinn an Greenpeace gespendet. 
\end{question}

\begin{question}{Haben Sie schon feste Vorstellungen, wie eine mögliche Koalition aussehen sollte? Welche Posten möchte die LSD gerne besetzen?}
    Unser Ziel ist es eine stabile Koalition für Goethopia im Sinne der sozialen Gerechtigkeit im Hinblick auf Gleichberechtigung der Geschlechter 
    und individueller Freiheit zu bilden. Damit die Ideen und Konzepte unserer potentiellen Koalitionspartner im Koalitionsvertrag zu erkennen sind, 
    sind wir durchaus kompromissbereit. Zwischen der LSD und der KKP sowie der FiP sehen wir einige Schnittmengen etwa in den Fragen Ehe für Alle, 
    der Durchsetzung von Internet, innerer Sicherheit, Eigentumsrechten und dem Mindestlohn. Für Letzteres steht auch die Partei MiG ein. 
\end{question}

\begin{question}{Ein linkes Bündnis aus EAP und KitKat wäre gleich stark wie die ihrige Partei. Sie buhlen beide um die „kleinen“ Parteien - Wie schätzen Sie ihre Chancen ein?}
    Wir sind davon überzeugt das unsere Partei das Mandat zur Bildung einer Regierung von den Bürgern erhalten hat und eine instabile Viererkonstellation von der 
    Mehrheit der Bürger nicht gewünscht ist. 
\end{question}

\begin{question}{Wie sehen die nächsten Schritte für ein Sondierungsgespräch aus?}
    Wir werden die kommenden Tage Kontakt zu KKP und MiG aufnehmen und Sondierungsgespräche führen.
\end{question}

\section{Verfassungsänderungen}
Der kursiv-gesetzte Text zeigt jeweils eine Änderung zur vorherigen Fassung an. Diese Änderung wurde vom Organisations-Komittee von Schule als Staat mehrheitlich beschlossen.
    \setcounter{articleno}{18}
    \begin{article}[Wahlrecht]
        \setcounter{enumi}{5}
    \item \textit{Um die Wahlergebnisse optimal abzubilden, kann die Parlamentsgröße auf mindestens 27 bzw. maximal 35 Sitze verändert werden.}
    \end{article}
    \setcounter{articleno}{31}
    \begin{article}[Öffentlichkeit und Unabhängigkeit der Justiz]
    \item Alle 6 Beamtinnen der Justiz werden vom Parlament ins Amt gewählt und können vom Parlament mit einer \textit{3/4} Mehrheit entmachtet werden.
    \end{article}
\end{document}