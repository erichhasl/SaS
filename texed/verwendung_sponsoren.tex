\documentclass{sasbase}

\usepackage[ngerman]{babel}
\usepackage{booktabs}

\begin{document}

\onecolumn
\title{Verwendung von Sponsorengeldern bei SaS}
\place{Ludwigsburg}
\datum{3. Juli 2018}
\edition{1}

\mytitle

\setlength{\parindent}{0mm}
\setlength{\parskip}{2mm}

\section{Finanzielles Konzept von Schule als Staat}

Grundsätzlich ist der Plan, dass unser Staat „Goethopia“ insgesamt Gewinn erzielt. Wichtig ist uns,
dass Betriebsleiter auf keinen Kosten sitzen bleiben müssen, wenn ihr Betrieb keine Gewinne erzielt.
Deshalb schießen wir den Betrieben nach Besprechung eines individuellen Wirtschaftsplans
einen Kredit vor, den diese im besten Fall auch zurückzahlen und darüber hinaus so viel einnehmen,
dass davon alle Mitarbeiter bezahlt werden können.

Zusätzlich muss unser Staat noch seine Beamten bezahlen, d.h. Parlamentarier, Staatsekretäre,
Minister etc., die bereits im Vorfeld des Projekts ehrenamtlich die Organisation vorantreiben.

Außerdem fallen dann noch kleinere Ausgaben an, wie Druckkosten der Staatswährung oder
Betriebskosten der Webseite.

Das genaue finanzielle Rahmenkonzept und einen vorraussichtlichen Haushalt ist dem Wirtschafts
Thesenpapier zu entnehmen.

\section{Risiken}

Das größte finanzielle Risko sind insgesamt die Betriebe. Wenn ein Betrieb Pleite geht, müssen die
Mitarbeiter anderweitig bezahlt und im Notfall vom Staat unterstützt werden. Zusätzlich
können dann die Kredite der Betriebe nicht mehr zurückgezahlt werden. Dieses Risiko wollen wir
teilweise durch die eingesammelten 10€ von jedem Teilnehmer absichern, brauchen aber noch zusätzlich
Gelder, um finanziell auf der sicheren Seite zu sein.

\section{Sponsoring}

Sponsorengelder verwenden wir besonders für die Kredite der Betriebe. Das heißt, wenn das Projekt
gut läuft, wird ein großer Teil dieser Gelder nicht benötigt werden, da er von den Betrieben
zurückgezahlt wird. Wie viel Geld am Ende übrig bleibt, können wir jetzt leider noch schwer
einschätzen. Insgesamt sind etwa 2500€ für die Kredite der Betriebe vorgesehen.

Es wäre optimal, wenn der Verein der Freunde und Ehemaligen des GGL e.V. uns bei der Finanzierung
der Kredite unterstützt.

Selbstverständlich können Sie entscheiden, wie mit dem Geld verfahren soll, das nicht benötigt
wurde.

\end{document}
