\documentclass[12pt]{article}

\usepackage[utf8]{inputenc}
\usepackage[german]{babel}
\usepackage{tikz}
\usepackage{extsizes}
\usepackage{geometry}
\usepackage{booktabs}
\usepackage{eurosym}

\geometry{
    margin=1.2cm
}

\newcommand{\kreis}{
    $\vcenter{\begin{tikzpicture}
        \draw (0,0) circle (0.2cm);
    \end{tikzpicture}}$
}

\setlength{\parindent}{0cm}
\pagenumbering{gobble}

\begin{document}

\textbf{\LARGE{Raumforderungen der Ministerien / Staatsbetriebe}}

\renewcommand{\arraystretch}{1.5}

\vspace{5mm}
\begin{tabular}{p{4cm}p{14cm}}
    \begin{tabular}{|p{.25\textwidth}|p{.5\textwidth}|p{.15\textwidth}|}
        \hline
        \textbf{Ministerium} & \textbf{Bedürfnisse} & \textbf{Raumwunsch} \\ \hline
        Wirtschaftsministerium & Leicht zugänglich, genug Platz für Warteschlange und Tische
        zum Abarbeiten & Aufenthaltsraum \\ \hline
        Parlament & Genug Platz für 32 Tische in leichter Rundung, Beamer & 302 \\ \hline
        & & \\ \hline
        & & \\ \hline
        & & \\ \hline
        & & \\ \hline
        & & \\ \hline
        & & \\ \hline
        & & \\ \hline
        & & \\ \hline
        & & \\ \hline
        & & \\ \hline
        & & \\ \hline
        & & \\ \hline
    \end{tabular} & \\
\end{tabular}
\end{document}
