\documentclass{sasbase}

\usepackage{lipsum}
\usepackage{enumitem}
\usepackage{graphicx}

\begin{document}

\title{Informationsblatt der APB}
\place{Ludwigsburg}
\datum{22. Januar 2018}
\edition{2}

\setcounter{secnumdepth}{5}

\mytitle

% OPTIONAL
%\squarestyle
% OR
\parensstyle

\section{Politische Bildung}
Das Informationsblatt beinhaltet auch immer FAQs zu neuen Gesetzen und der Funktionsweise der staatlichen Organe.
Fragen k\"{o}nnen gerne jederzeit an den Ausschuss f\"{u}r politische Bildung gesendet werden, diese werden so fr\"{u}h als m\"{o}gliche bearbeitet und in der n\"{a}chsten Ausgabe beantwortet.

\topic{Wahl}

\begin{question}{Wie genau läuft die Wahl ab?}
    Am Donnerstag, den 1. Februar werden eifrige Wahlhelfer in der 3./4. Stunde alle Klassen besuchen, die Wahlzettel austeilen,
    ausfüllen lassen und wieder einsammeln. Dabei werden sowohl die Präsidentin als auch die
    Parteien für das Parlament gewählt.
    \\
    \\
    \noindent Wer an diesem Tag \textbf{nicht} anwesend ist, hat aufgrund organisatorischer Einschränkungen \textbf{keine} andere Möglichkeit zu wählen.
    \\
    \\
    
    \noindent Sobald die Wahlzettel ausgezählt sind, werden die Wahlergebnisse per Durchsage und per
    Ankündigungsblatt durchgegeben.
\end{question}

\begin{question}{Wer steht überhaupt zur Wahl?}
    \textbf{Parteien für die Parlamentswahl}
    \begin{itemize}
        \item Partei für Gleichberechtigung \textbf{(PfG)}
        \item Goethopische Gerechtigkeitspartei \textbf{(GGP)}
        \item Minderheitengremium \textbf{(MIG)}
        \item Kleine Kätzchen Partei \textbf{(KKP)}
        \item Einheitliche Arbeiterpartei \textbf{(EAP)}
        \item Liberale Sozialdemokraten \textbf{(LSD)}
        \item Kommunistisch Altruistische Partei \texbf{(KitKat)}
        \item Die Blauen \textbf{(DB)}
        \item Die Goethopia Partei \textbf{(DGP)}
        \item Frauen im Parlament \textbf{(FiP)}
    \end{itemize}

    \newpage
    \textbf{Präsidentschaftskandidaten}
    \begin{itemize}
        \item Aylin \textbf{Bozcali}
        \item Paula-Francesca \textbf{Pompei}
        \item David \textbf{Schwarz}
        \item Eric \textbf{Reischl}
    \end{itemize}

    Auf der Rückseite der Schule als Staat Stellwand sind auch Wahlplakate aufgehängt.
\end{question}

\begin{question}{Wo kann ich mich informieren?}
    Um genauere Informationen über die Parteien und die Präsidentschaftskandidaten zu erhalten,
    geht einfach auf die Goethopia Webseite:
    \\\\
    \noindent\textbf{www.goethopia.de/wahl}
    \\\\
    Ihr könnt natürlich auch die Parteivorsitzenden oder Präsidentschaftskandidaten direkt
    ansprechen, um euch genauer über deren Ziele zu informieren. Darüber hinaus machen einige Parteien auch schon
    eifrig Wahlwerbung und es hängen Wahlplakate im Schulhaus aus. 
\end{question}

\begin{question}{Wie wähle ich?}
    Ein Demo-Stimmzettel hängt hier an diesem Brett aus. An sich ist das Ganze intuitiv, du machst ein Kreuz auf der linken Seite für
    den Präsidenten und ein Kreuz auf der rechten Seite für deine Partei. 
    Wenn nicht \textbf{in jeder Spalte exakt ein Kreuz ist}, ist der Wahlzettel ungültig.
    Siehe dazu auch die folgende Graphik:

\topic{Nach der Wahl}

\begin{question}{Ab wann tagt das Parlament?}
    Sobald das endgültige Wahlergebnis veröffentlicht wurde, haben die Parteien Zeit, um eine Regierung zu bilden. 
    In der Woche vom 19.02. bis 23.02. muss die Präsidentin das Parlament das erste Mal einberufen, damit sich das Parlament konstituiert.
    Diese erste Sitzung wird als Tagesordnung vor allem formelle Aspekte wie Ratifizierung der Geschäftsordnung, Besetzung der Ministerien und Einstellungen wichtiger Beamten haben, 
    ab der zweiten Sitzung werden dann Gesetze erlassen und weitere Beamte eingestellt.
    \\
    \\
    Das Parlament entscheidet selber, wann es tagt, von der Häufigkeit der Parlamentssitzungen hängt allerdings der Erfolg des gesamten Projekts ab, deshalb
    ist das Parlament angehalten, wöchentlich zu tagen.
\end{question}
\begin{question}{Was macht das Parlament?}
\end{question}

\section{Verfassungsänderungen}

\section{Impressum}
\begin{minipage}{0.4\linewidth}
\includegraphics[width=\textwidth]{apb_icon.png}
\end{minipage}
\begin{minipage}{0.5\linewidth}
{\raggedright stellvertretend Christian Merten und Nils Hebach.}
\end{minipage}
\end{document}
