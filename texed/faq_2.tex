\documentclass{sasbase}

\usepackage{lipsum}
\usepackage{enumitem}
\usepackage{graphicx}

\begin{document}

\title{Informationsblatt der APB}
\place{Ludwigsburg}
\datum{22. Januar 2018}
\edition{2}

\setcounter{secnumdepth}{5}

\mytitle

% OPTIONAL
%\squarestyle
% OR
\parensstyle

\section{Politische Bildung}
Das Informationsblatt beinhaltet auch immer FAQs zu neuen Gesetzen und der Funktionsweise der staatlichen Organe.
Fragen k\"{o}nnen gerne jederzeit an den Ausschuss f\"{u}r politische Bildung gesendet werden, diese werden so fr\"{u}h als m\"{o}gliche bearbeitet und in der n\"{a}chsten Ausgabe beantwortet.

\topic{Wahl}

\begin{question}{Wie genau läuft die Wahl ab?}
    Am 1. Februar werden eifrige Wahlhelfer alle Klassen besuchen, die Wahlzettel austeilen,
    ausfüllen lassen und wieder einsammeln. Dabei werden sowohl der/die Präsidentin als auch die
    Parteien für das Parlament gewählt.
    \\
    \\
    \noindent\textbf{Wer am 1. Februar nicht da ist, kann nicht wählen!}
    \\
    
    \noindent Sobald die Wahlzettel ausgezählt sind, werden die Wahlergebnisse per Durchsage und per
    Ankündigungsblatt durchgegeben.
\end{question}

\begin{question}{Wer steht überhaupt zur Wahl?}
    \textbf{Parteien für die Parlamentswahl}
    \begin{itemize}
        \item Partei für Gleichberechtigung (PfG)
        \item Goethopische Gerechtigkeitspartei (GGP)
        \item Minderheitengremium (MIG)
        \item Kleine Kätzchen Partei (KKP)
        \item Einheitliche Arbeiterpartei (EAP)
        \item Liberale Sozialdemokraten (LSD)
        \item Kommunistisch Altruistische Partei (Kitkat)
        \item Die Blauen (DB)
        \item Die Goethopia Partei (DGP)
        \item Frauen im Parlament (FiP)
    \end{itemize}

    \newpage
    \textbf{Präsidentschaftskandidaten}
    \begin{itemize}
        \item Aylin Bozcali
        \item Paula-Francesca Pompei
        \item David Schwarz
        \item Eric Reischl
    \end{itemize}

    Auf der Rückseite der Schule als Staat Stellwand sind auch Wahlplakate aufgehängt.
\end{question}

\begin{question}{Wer? Wie? Wo?}
    Um genauere Informationen über die Parteien und die Präsidentschaftskandidaten zu erhalten,
    geht einfach auf die Goethopia Webseite:
    \\\\
    \noindent\textbf{www.goethopia.de/wahl}
    \\\\
    Ihr könnt natürlich auch die Parteivorsitzenden oder Präsidentschaftskandidaten direkt
    ansprechen, um euch genauer über deren Ziele zu informieren.
\end{question}

\begin{question}{Was macht das Parlament?}
\end{question}


\section{Impressum}
\begin{minipage}{0.4\linewidth}
\includegraphics[width=\textwidth]{apb_icon.png}
\end{minipage}
\begin{minipage}{0.5\linewidth}
{\raggedright stellvertretend Christian Merten und Nils Hebach.}
\end{minipage}
\end{document}
