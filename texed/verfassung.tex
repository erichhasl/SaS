\documentclass{sasbase}

\usepackage{lipsum}
\usepackage{enumitem}
\usepackage{graphicx}

\begin{document}

\title{Verfassung Goethopias}
\place{Ludwigsburg}
\datum{17. November 2017}
\edition{1}

\setcounter{secnumdepth}{5}

\mytitle

% OPTIONAL
%\squarestyle
% OR
\parensstyle

\section{Verfassung von Goethopia}

\segmentoflaw{Präambel}

Die vorliegende Verfassung wurde am 06. November 2017 mehrheitlich von dem Schule-als-Staat Komitee verabschiedet und tritt hiermit in Kraft. Vorgesehen sind weitere Anpassungen der Verfassung durch ein sich in Zukunft konstituierendes Parlament, die auch \"{u}ber diesen Weg ver\"{o}ffentlicht werden.\\
\indent Wenn im Folgenden die weibliche Form verwendet wird, so ist die männliche Form immer mit eingeschlossen. Die Beschränkung auf ein Geschlecht beruht nur auf besserer Lesbarkeit.

\segmentoflaw{Grundrechte}

\begin{article}
	\item Die Würde des Menschen ist unantastbar. Sie zu achten und zu schützen ist Verpflichtung aller staatlichen Gewalt.
 	\item Das Volk Goethopias bekennt sich darum zu unverletzlichen und unveräußerlichen Menschenrechten als Grundlage jeder menschlichen Gemeinschaft, des Friedens und der Gerechtigkeit in der Welt.
	\item Die nachfolgenden Grundrechte binden Gesetzgebung, vollziehende Gewalt und Rechtsprechung als unmittelbar geltendes Recht.
\end{article}

\begin{article}
	\item Jede hat das Recht auf die freie Entfaltung ihrer Persönlichkeit, soweit sie nicht die Rechte anderer verletzt und nicht gegen die verfassungsmäßige Ordnung oder das Sittengesetz verstößt.	
	\item Jede hat das Recht auf Leben und körperliche Unversehrtheit. Die Freiheit der Person ist unverletzlich. In diese Rechte darf nur auf Grund eines Gesetzes eingegriffen werden.
\end{article}

\begin{article}
	\item Alle Menschen sind vor dem Gesetz gleich.
	\item Frauen und Männer sind gleichberechtigt. Der Staat fördert die tatsächliche Durchsetzung der Gleichberechtigung von Frauen und Männern und wirkt auf die Beseitigung bestehender Nachteile hin.
	\item Niemand darf wegen ihres Geschlechtes, ihrer Abstammung, ihrer Rasse, ihrer Sprache, ihrer Heimat und Herkunft, ihres Glaubens, ihrer religiösen oder politischen Anschauungen und ihrer sexuellen Orientierung benachteiligt oder bevorzugt werden. Niemand darf wegen ihrer Behinderung benachteiligt werden.
\end{article}

\begin{article}[Religionsfreiheit]
	\item Die Freiheit des Glaubens, des Gewissens und die Freiheit des religiösen und weltanschaulichen Bekenntnisses sind unverletzlich.	
	\item Die ungestörte Religionsausübung wird gewährleistet.
\end{article}

\begin{article}[Meinungsfreiheit]
	\item Jeder hat das Recht, seine Meinung in Wort, Schrift und Bild frei zu äußern und zu verbreiten und sich aus allgemein zugänglichen Quellen ungehindert zu unterrichten. Die Pressefreiheit und die Freiheit der Berichterstattung durch Rundfunk und Film werden gewährleistet. Eine Zensur findet nicht statt.
	\item Diese Rechte finden ihre Schranken in den Vorschriften der allgemeinen Gesetze, den gesetzlichen Bestimmungen zum Schutze der Jugend und in dem Recht der persönlichen Ehre.
	\item Kunst und Wissenschaft, Forschung und Lehre sind frei. Die Freiheit der Lehre entbindet nicht von der Treue zur Verfassung.
\end{article}

\begin{article}[Ehe und Familie]
	\item Ehe und Familie stehen unter dem besonderen Schutze der staatlichen Ordnung.
	\item Pflege und Erziehung der Kinder sind das natürliche Recht der Eltern und die zuvörderst ihnen obliegende Pflicht. Über ihre Betätigung wacht die staatliche Gemeinschaft.
	\item Gegen den Willen der Erziehungsberechtigten dürfen Kinder nur auf Grund eines Gesetzes von der Familie getrennt werden, wenn die Erziehungsberechtigten versagen oder wenn die Kinder aus anderen Gründen zu verwahrlosen drohen.
	\item Jede Mutter hat Anspruch auf den Schutz und die Fürsorge der Gemeinschaft.
	\item Den unehelichen Kindern sind durch die Gesetzgebung die gleichen Bedingungen für ihre leibliche und seelische Entwicklung und ihre Stellung in der Gesellschaft zu schaffen wie den ehelichen Kindern.
\end{article}

\begin{article}[Versammlungsfreiheit]
	\item Alle B\"{u}rgerinnen Goethopias haben das Recht, sich ohne Anmeldung oder Erlaubnis friedlich und ohne Waffen zu versammeln.
	\item Für Versammlungen unter freiem Himmel kann dieses Recht durch Gesetz oder auf Grund eines Gesetzes beschränkt werden.
\end{article}

\begin{article}[Vereinsfreiheit]
	\item Alle B\"{u}rgerinnen Goethopias haben das Recht, Vereine und Gesellschaften zu bilden.
	\item Vereinigungen, deren Zwecke oder deren Tätigkeit den Strafgesetzen zuwiderlaufen oder die sich gegen die verfassungsmäßige Ordnung oder gegen den Gedanken der Völkerverständigung richten, sind verboten.
	\item Das Recht, zur Wahrung und Förderung der Arbeits- und Wirtschaftsbedingungen Vereinigungen zu bilden, ist für jede Frau und für alle Berufe gewährleistet. Abreden, die dieses Recht einschränken oder zu behindern suchen, sind nichtig, hierauf gerichtete Maßnahmen sind rechtswidrig.
\end{article}

\begin{article}[Briefgeheimnis]
	\item Das Briefgeheimnis sowie das Post- und Fernmeldegeheimnis sind unverletzlich.
	\item Beschränkungen dürfen nur auf Grund eines Gesetzes angeordnet werden. Dient die Beschränkung dem Schutze der freiheitlichen demokratischen Grundordnung oder des Bestandes oder der Sicherung des Bundes oder eines Landes, so kann das Gesetz bestimmen, daß sie dem Betroffenen nicht mitgeteilt wird und daß an die Stelle des Rechtsweges die Nachprüfung durch von der Volksvertretung bestellte Organe und Hilfsorgane tritt.
\end{article}

\begin{article}[Innerstaatliche Freiz\"{u}gigkeit]
	\item Alle B\"{u}rgerinnen Goethopias genießen Freizügigkeit im ganzen Staatsgebiet.
	\item Dieses Recht darf nur durch Gesetz oder auf Grund eines Gesetzes und nur für die Fälle eingeschränkt werden, in denen eine ausreichende Lebensgrundlage nicht vorhanden ist und der Allgemeinheit daraus besondere Lasten entstehen würden oder in denen es zur Abwehr einer drohenden Gefahr für den Bestand oder die freiheitliche demokratische Grundordnung des Bundes oder eines Landes, zur Bekämpfung von Seuchengefahr, Naturkatastrophen oder besonders schweren Unglücksfällen, zum Schutze der Jugend vor Verwahrlosung oder um strafbaren Handlungen vorzubeugen, erforderlich ist.
\end{article}

\begin{article}[Freiheit der Berufswahl]
	\item Alle B\"{u}rgerinnen Goethopias haben das Recht, Beruf, Arbeitsplatz und Ausbildungsstätte frei zu wählen. Die Berufsausübung kann durch Gesetz oder auf Grund eines Gesetzes geregelt werden.	
	\item Niemand darf zu einer bestimmten Arbeit gezwungen werden, außer im Rahmen einer herkömmlichen allgemeinen, für alle gleichen öffentlichen Dienstleistungspflicht.
	\item Zwangsarbeit ist nur bei einer gerichtlich angeordneten Freiheitsentziehung zulässig.
\end{article}

\begin{article}[Privatsph\"{a}re]
	\item Die Wohnung ist unverletzlich.
	\item Durchsuchungen dürfen nur durch die Richterin, bei Gefahr im Verzuge auch durch die in den Gesetzen vorgesehenen anderen Organe angeordnet und nur in der dort vorgeschriebenen Form durchgeführt werden.
	\item Begründen bestimmte Tatsachen den Verdacht, daß jemand eine durch Gesetz einzeln bestimmte besonders schwere Straftat begangen hat, so dürfen zur Verfolgung der Tat auf Grund richterlicher Anordnung technische Mittel zur akustischen Überwachung von Wohnungen, in denen die Beschuldigte sich vermutlich aufhält, eingesetzt werden, wenn die Erforschung des Sachverhalts auf andere Weise unverhältnismäßig erschwert oder aussichtslos wäre. Die Maßnahme ist zu befristen. Die Anordnung erfolgt durch das Strafgericht. Bei Gefahr im Verzuge kann sie auch durch eine einzelne Richterin getroffen werden.
	\item Zur Abwehr dringender Gefahren für die öffentliche Sicherheit, insbesondere einer gemeinen Gefahr oder einer Lebensgefahr, dürfen technische Mittel zur Überwachung von Wohnungen nur auf Grund richterlicher Anordnung eingesetzt werden. Bei Gefahr im Verzuge kann die Maßnahme auch durch eine andere gesetzlich bestimmte Stelle angeordnet werden; eine richterliche Entscheidung ist unverzüglich nachzuholen.
	\item Sind technische Mittel ausschließlich zum Schutze der bei einem Einsatz in Wohnungen tätigen Personen vorgesehen, kann die Maßnahme durch eine gesetzlich bestimmte Stelle angeordnet werden. Eine anderweitige Verwertung der hierbei erlangten Erkenntnisse ist nur zum Zwecke der Strafverfolgung oder der Gefahrenabwehr und nur zulässig, wenn zuvor die Rechtmäßigkeit der Maßnahme richterlich festgestellt ist; bei Gefahr im Verzuge ist die richterliche Entscheidung unverzüglich nachzuholen.
	\item Eingriffe und Beschränkungen dürfen im übrigen nur zur Abwehr einer gemeinen Gefahr oder einer Lebensgefahr für einzelne Personen, auf Grund eines Gesetzes auch zur Verhütung dringender Gefahren für die öffentliche Sicherheit und Ordnung, insbesondere zur Behebung der Raumnot, zur Bekämpfung von Seuchengefahr oder zum Schutze gefährdeter Jugendlicher vorgenommen werden.
\end{article}

\begin{article}[Recht auf Eigentum]
	\item Das Eigentum und das Erbrecht werden gewährleistet. Inhalt und Schranken werden durch die Gesetze bestimmt.
	\item Eigentum verpflichtet. Sein Gebrauch soll zugleich dem Wohle der Allgemeinheit dienen.
	\item Eine Enteignung ist nur zum Wohle der Allgemeinheit zulässig. Sie darf nur durch Gesetz oder auf Grund eines Gesetzes erfolgen, das Art und Ausmaß der Entschädigung regelt. Die Entschädigung ist unter gerechter Abwägung der Interessen der Allgemeinheit und der Beteiligten zu bestimmen. Wegen der Höhe der Entschädigung steht im Streitfalle der Rechtsweg vor den ordentlichen Gerichten offen.
\end{article}

\begin{article}[Demokratiegrundlage]
	\item Jede hat das Recht, sich einzeln oder in Gemeinschaft mit anderen schriftlich mit Bitten oder Beschwerden an die zuständigen Stellen und an die Volksvertretung zu wenden.
\end{article}

\begin{article}[Einschr\"{a}nkung der Grundrechte]
	\item Wer die Freiheit der Meinungsäußerung, insbesondere die Pressefreiheit, die Versammlungsfreiheit, die Vereinigungsfreiheit , das Brief-, Post- und Fernmeldegeheimnis, das Eigentum oder das Asylrecht zum Kampfe gegen die freiheitliche demokratische Grundordnung mißbraucht, verwirkt diese Grundrechte. Die Verwirkung und ihr Ausmaß werden durch das Verfassungsgericht ausgesprochen.
\end{article}

\segmentoflaw{Der Staat}

\begin{article}[Staatsform]
	\item Goethopia ist ein sozialer und demokratischer Staat.
	\item Alle Staatsgewalt geht vom Volke aus. Sie wird vom Volke in Wahlen und Abstimmungen und durch besondere Organe der Gesetzgebung, der vollziehenden Gewalt und der Rechtsprechung ausgeübt.
	\item Die Gesetzgebung ist an die verfassungsmäßige Ordnung, die vollziehende Gewalt und die Rechtsprechung sind an Gesetz und Recht gebunden.
	\item Gegen jede, die es unternimmt, diese Ordnung zu beseitigen, haben alle B\"{u}rgerinnen das Recht zum Widerstand, wenn andere Abhilfe nicht möglich ist.
\end{article}

\begin{article}[Staatsangehörigkeit]
	\item Staatsangehörige ist, wer Schülerin, Lehrerin oder andersweitig Angestellte des Goethe-Gymnasium Ludwigsburg ist.
	\item Die Staatsangehörigkeit kann von Nicht-Staatsbürgerinnen nur erlangt werden, wenn dafür ein besonderer Grund besteht.
	\item Alle Staantsangehörigen unterliegen einzig und allein der Gesetzbarkeit des Staates Goethopia, solange sie sich auf goethopischem Staatsgebiet befinden.
	\item Absatz 3 gilt nicht, wenn gleichzeitig Gesetze des deutschen Staates verletzt werden.
\end{article}

\begin{article}[Staatsgebiet]
	\item Goethopisches Gebiet ist dem Gebäudekomplex des Goethe-Gymnasiums Ludwigsburg Zugehöriges.
	\item Durch den Verkauf von staatlichem Gebiet unterliegt dieses immer noch dem Staat, allein Besitz- und Baurechte gehen auf den Käufer über.  
\end{article}

\begin{article}[Wahlrecht]
	\item Alle Staatsbürgerinnen sind wahlberechtigt.
	\item Absatz 1 kann durch Artikel 35 der Verfassung für Einzelfälle außer Kraft gesetzt werden.
	\item Wahlberechtigte wählen Parteien, keine Einzelpersonen.
	\item Eine Partei muss mindestens 10 Mitglieder und 20 Unterschriften vorweisen können, um zur Wahl anzutreten. Es gilt das Ausschlie{\ss}lichkeitsprinzip, das hei{\ss}t eine B\"{u}rgerin darf nur Mitglied bei einer Partei sein. Von den 10 ben\"{o}tigten Parteimitgliedern muss mindestens ein Mitglied aus den Klassenstufen 5-7, ein weiteres Mitglied aus den Klassenstufen 8-9 und ein drittes Mitglied aus den Klassenstufen 10-12 kommen.
	\item Die Anzahl der Parlaments-Abgeordneten (31, davon 1 Parlamentspräsidentin) wird nun verhältnismäßig unter den gewählten Parteien verteilt, die mindestens 5\% der abgegebenen Stimmen erreicht haben.
	\item Parteien haben vor der Wahl Listen offenzulegen, aus denen sich erschließt, welche Abgeordnete in welcher Reihenfolge in das Parlament einziehen.
\end{article}

\begin{article}[Präsidentin]
	\item Das repräsentative Staatsoberhaupt Goethopias ist die Präsidentin.
	\item Die Präsidentin übernimmt keine Verantwortung für das Handeln der Regierung.
	\item Per Mehrheitswahl wird sie direkt von allen wahlberechtigten Bürgerinnen gewahlt.
	\item Jede Staatsbürgerin über 12 Jahren kann sich zur Präsidentin aufstellen lassen, sofern sie die nötige Unterstützung von mindestens 10 weiteren Staatsbürgerinnen vorweisen kann.
	\item Alle vom Parlament beschlossenen Gesetze müssen innerhalb des darauffolgenden Tages unterzeichnet und damit ratifiziert werden, damit diese Gültigkeit erlangen. Falls sie die Rechtmaßigkeit der Gesetze bezweifelt, besteht die Moglichkeit, das Verfassungsgericht zu informieren.
\end{article}

\segmentoflaw{Die Legislative}

\begin{article}[Parlamentsgrunds\"{a}tze]
	\item Die Abgeordnete des Parlaments werden in allgemeiner, unmittelbarer, freier, gleicher und geheimer Wahl gewählt.
	\item Das Parlament verhandelt öffentlich. Auf Antrag eines Zehntels seiner Mitgliederinnen oder auf Antrag der Regierung kann mit Zweidrittelmehrheit die Öffentlichkeit ausgeschlossen werden. Über den Antrag wird in nichtöffentlicher Sitzung entschieden.
	\item Zu einem Beschlusse des Parlament ist die Mehrheit der abgegebenen Stimmen erforderlich, soweit dieses Grundgesetz nichts anderes bestimmt.
	\item Wahrheitsgetreue Berichte über die öffentlichen Sitzungen des Bundestages und seiner Ausschüsse bleiben von jeder Verantwortlichkeit frei.
	\item Dem Parlament sitzt eine Parlamentspräsidentin vor, die aus dem Kreis der Abgeordneten benannt wird, die Diskussionen leitet und darauf achtet, dass die Geschäftsordnung eingehalten wird. Sie wird von dem gesamten Parlament gewählt und ist eine Legislaturperiode im Amt.\end{article}

\begin{article}[Aufgaben des Parlamentes]
	\item Das Parlament erlässt allgemein gültige Gesetze.
	\item Die Regierung wird vom Parlament kontrolliert.
	\item Das Parlament bildet Ausschüsse zur Untersuchung von Sachthemen. Näheres regelt die Geschaftsordnung.
	\item Der Staatshaushalt wird vom Parlament beschlossen.
	\item Die Kanzlerin, Richterinnen und die Polizeivorsteherin werden durch das Parlament gewählt.
\end{article}

\begin{article}[Abgeordnete]
	\item Abgeordnete sind nicht an Aufträge oder Weisungen gebunden und lediglich ihrem Gewissen unterworfen.
	\item Alle Abgeordneten des Parlaments haben eine Amtsimmunität, dies gilt nicht für verleumderische Beleidigungen. Die Immunität kann jedoch jederzeit vom Verfassungsgericht aufgehoben werden.
\end{article}

\begin{article}[Beziehung zwischen Regierung und Parlament]
	\item Das Parlament und seine Ausschüsse können die Anwesenheit jedes Mitgliedes der Regierung verlangen.
	\item Die Mitglieder der Regierung sowie ihre Beauftragten haben zu allen Sitzungen des Parlaments und seiner Ausschüsse Zutritt. Sie müssen jederzeit gehört werden.
\end{article}

\begin{article}[Gesch\"{a}ftsordnung]
	\item Zu Beginn der Legislaturperiode wird vom Parlament eine Geschäftsordnung festgelegt, die die ganze Legislaturperiode über Gültigkeit hat.
	\item Die Parlamentspräsidentin achtet auf Einhaltung dieser Geschäftsordnung.
	\item Gesetzesvorlagen werden beim Parlament durch die Regierung oder aus der Mitte des Parlaments, unterstützt von mindestens 5 Abgeordneten, eingebracht. Eingebrachte Vorlagen müssen im Plenum diskutiert und als Gesetze erlassen oder zurückgewiesen werden.
	\item Die Parlamentspräsidentin legt die Tagesordnung und die Reihenfolge, in der ausstehende Gesetzesvorlagen bearbeitet werden müssen, fest. 3 Abgeordnete können gemeinsam eine Änderungen der Tagesordnung beantragen, über die das Plenum mit einfachem Mehrheitsbeschluss abzustimmen hat.
\end{article}

\segmentoflaw{Die Exekutive}

\begin{article}[Kanzlerin]
	\item Die Kanzlerin wird durch das Parlament gewählt.
	\item Sie leitet die Regierung und sitzt diesem vor. Ihr obliegt die Richtlinenkompetenz und sie trägt die Regierungsverantwortung gegenüber dem Parlament.
	\item Die Kanzlerin ernennt eine Stellvertreterin.
\end{article}

\begin{article}[Regierung]
	\item Die Regierung besteht aus den Ministerinnen (Innenministerin, Außenministerin, Wirtschaftsministerin, Arbeitsministerin und Kultusministerin) und der Kanzlerin. Die Ministerinnen werden von der Kanzlerin vorgeschlagen und durch die Präsidentin ernannt. Sie müssen nicht Teil des gewählten Parlaments sein.
	\item Innerhalb der von der Bundeskanzlerin bestimmten Richtlinien leitet jede Ministerin seinen Geschäftsbereich selbstständig und unter eigener Verantwortung.
	\item Die Maximalgröße pro Ministerium beträgt 4 Mitarbeiterinnen (3 Beamtinnen, 1 Ministerin).
\end{article}

\begin{article}[Staatsanwaltschaft]
	\item Die Staatsanwaltschaft untersteht nicht der Regierung und wird nicht von den Parteien gestellt, sondern von dem Parlament per Mehrheitsbeschluss gewählt.
	\item Die Saatsanwaltschaft ist der Dreh- und Angelpunkt des Strafverfahrens. Sie hat beim Verdacht von Straftaten i. d. R. von Amts wegen nach dem Legalitätsprinzip einzuschreiten; sie hat Anzeigen von Straftaten entgegenzunehmen und mit der Polizei, die Hilfsorgan der Staatsanwaltschaft ist, oder den Gerichten den Sachverhalt zu untersuchen.
	\item Die Staatsanwaltschaft hat nicht nur die belastenden, sondern auch die entlastenden Umstände zu ermitteln und hervorzuheben.
	\item Der Staatsanwaltschaft steht das Recht zu, bestimmte Zwangsmaßnahmen (z. B. vorläufige Festnahme, Beschlagnahme, Durchsuchung, Untersuchungshaft) entweder (bei Gefahr im Verzug) selbst anzuordnen oder sie beim zuständigen Richter zu beantragen.
	\item Der Staatsanwaltschaft obliegt die Entscheidung darüber, ob das Verfahren einzustellen oder bei hinreichendem Tatverdacht Anklage zu erheben ist (Opportunitätsprinzip). In der Hauptverhandlung vertritt sie die Anklage; sie kann – auch zugunsten des Angeklagten – Rechtsmittel einlegen.
	\item Die Staatsanwaltschaft ist außerdem Vollstreckungsbehörde (Strafvollstreckung).
\end{article}

\begin{article}[Polizei und Gewaltenmonopol]
	\item Das Parlament bestimmt eine Polizeivorsteherin. Die Polizei untersteht nicht dem Innenministerium, sondern ist eine unabhängige Instanz der Exekutive.
	\item Die Polizei hat repräsentativ die vom Staat wahrgenommene ausschließliche Befugnis, auf seinem Staatsgebiet physische Gewalt einzusetzen oder ihren Einsatz zuzulassen.	\item Privatpersonen dürfen physische Gewalt nur aufgrund staatlicher Ermächtigung oder Delegation ausüben, wenn staatliche Gewalt ihre Schutzaufgabe nicht rechtzeitig wahrnehmen kann.
	\item Das Gewaltenmonopol soll für den Bürger im Verhältnis zu den Mitbürgern freiheitssichernd wirken.
\end{article}

\segmentoflaw{Judikative}

\begin{article}[Verfassungsgericht]
	\item Das Verfassungsgericht besteht aus 3 hauptberuflichen Richterinnen.
	\item Es kann auch ohne Klage vom Parlament verabschiedete Gesetze jederzeit einstimmig als verfassungswidrig zurückweisen oder im Eilverfahren stoppen.
	\item Gleichzeitig hat das Verfassungsgericht über nicht in der Verfassung oder im Strafgesetzbuch enthaltene Klagen und Vorfälle zu entscheiden und muss über Revisionen des Zivil- und Strafgerichts entscheiden.
	\item Einstimmig kann das Verfassungsgericht Klagen abweisen und Urteile fällen.
	\item Das Verfassungsgericht kann einstimmig die Immunität eines Abgeordneten aufheben.
\end{article}

\begin{article}[Zivil- und Strafgericht]
	\item Das Zivil- und Strafgericht besteht aus 5 Richterinnenn, davon sind 3 hauptberuflich und 2 für jeden Tag am Anbeginn der Zeit aus Schöffenbewerbungen ausgewählt.
	\item Es ist zuständig für alle Klagen von Privatpersonen gegen andere Privatpersonen oder öffentliche Instanzen.
	\item Es kann Urteile mit einer einfachen Mehrheit beschließen und Klagen mit 4 von 5 Stimmen abweisen.
	\item Das Gericht kann nicht ohne Klage aktiv werden.
\end{article}

\begin{article}[Öffentlichkeit und Unabh\"{a}ngigkeit der Justiz]
	\item Alle 6 Beamtinnen der Justiz werden vom Parlament ins Amt gewählt und können vom Parlament mit einer 2/3 Mehrheit entmachtet werden.
	\item Die Urteilsbegründung muss öffentlich vorliegen.
\end{article}

\segmentoflaw{Die Verfassung}

\begin{article}[Ewigkeitsklausel]
	\item Eine Änderung dieser Verfassung, durch welche die Gliederung des Staates oder die in den Artikeln 1, 2,3, 16 und 33 niedergelegten Grundsätze berührt werden, ist unzulässig.
\end{article}

\begin{article}[Verfassungsänderung und -einhaltung]
	\item Die gesamten Handlungen von Einzelpersonen und Personenverbänden müssen zu jedem Zeitpunkt verfassungskonform sein. Für die Einhaltung der Verfassung ist auf exekutiver Seite die Staatsanwaltschaft und die Polizei zuständig, auf judikativer Seite das Verfassungsgericht.
	\item Verfassungsänderungen müssen von einer einfachen Mehrheit der wahlberechtigten Bevölkerung bestätigt werden.
	\item Bei Verfassungsänderungen, die das Wesen dieser nicht verändern, kann diese mit einer 3/4 Mehrheit des Parlaments und Zustimmung des Verfassungsgericht geändert werden. Das Verfassungsgericht entscheidet im Zweifelsfall, ob das Wesen verändert werden würde.
\end{article}

\begin{article}[Einschränkung der Verfassung]
	\item Ist das Interesse der öffentlichen Sicherheit und offentlichen Ruhe schwerwiegender als das Recht der Meinungsfreiheit des Einzelnen oder liegt eine Gefährdung der öffentlichen Ordnung vor, darf das Verfassungsgericht Parteien oder Organisationen verbieten.
	\item Im Extremfall kann dem Einzelnen das Stimmrecht entzogen werden, wenn er nachweislich den Staat als solchen stürzen will.
	\item Wer die Freiheit der Meinungsäußerung, insbesondere die Pressefreiheit, die Versammlungsfreiheit, die Vereinigungsfreiheit , das Brief-, Post- und Fernmeldegeheimnis, das Eigentum oder das Asylrecht zum Kampfe gegen die freiheitliche demokratische Grundordnung mißbraucht, verwirkt diese Grundrechte. Die Verwirkung und ihr Ausmaß werden durch das Verfassungsgericht ausgesprochen.
\end{article}
\newpage

\section{Impressum}
\begin{minipage}[t]{0.15\textwidth}
\includegraphics[width=\textwidth]{apb_icon.png}
\end{minipage}
\begin{minipage}[t]{0.15\textwidth}
stellvertretend Christian Merten und Nils Hebach.
\end{minipage}
\end{document}
