\documentclass{sasbase}

\usepackage[ngerman]{babel}
\usepackage{booktabs}

\begin{document}

\onecolumn
\title{Wirtschaft - Thesenpapier}
\place{Ludwigsburg}
\datum{11. Dezember 2017}
\edition{1}

\mytitle

\setlength{\parindent}{0mm}
\setlength{\parskip}{2mm}

\section{System}

Die Wirtschaft Goethopias sieht Betriebe vor, die von Schüler oder Lehrern geführt bzw. gegründet
werden können.
Alle Schüler und Lehrer müssen dabei eine Schicht arbeiten, während der anderen jedoch trotzdem
anwesend sein, um in dieser Zeit Ausgaben zu tätigen und damit wiederum das Wirtschaftssystem
ankurbeln.

Aktuell sehen die Planungen \textbf{4 Tage} vor, mit jeweils zwei Schichten à \textbf{3 Stunden}.

Ausnahmefall sind verschiedene Beamtenstellen, die vom Staat bezahlt werden (Abgeordnete,
Funktionsträger, etc) und nicht in zwei Schichten aufgeteilt werden, da diese meist gewählt sind
und diese Rolle dann auch für den gesamten Zeitraum aus- und erfüllen sollen / dürfen.

Zu Beginn sollen von jedem Bürger (Schüler und Lehrer) 10€ eingesammelt werden, davon
werden 4€ behalten, um die Kredite für die Betriebe zu decken, der Rest wird an die Schüler
ausgezahlt in G-Mark.\\
Am Ende des Projekts werden jedem Bürger die 10€ wieder zurück gezahlt, vorausgesetzt die Kosten
werden gedeckt, was bei einigermaßen sinnvollem Wirtschaften der Betriebe aber machbar ist.

Es wird am Ende außer den 10€ kein Geld an die Bürger ausgezahlt, d.h. das Bunkern von Geld ist
nicht lukrativ (sonst kollabiert das Wirtschaftssystem).\\
Der genaue Verwendungszweck der potentiellen Gewinne wird vom Parlament im Vorfeld des Projekts
festgelegt.

\section{Betriebe}

Jeder Betrieb muss zur Gründung einen Wirtschaftsplan vorlegen, aus dem hervorgeht, wie genau
die Planung aussieht. Je nach Perspektive werden die Betriebe dann zugelassen oder abgelehnt.

Nach eingehender Prüfung können die Betriebe dann einen \textbf{Kredit} beanspruchen, der in G-Mark
aufgeschrieben, aber in Euro ausgezahlt wird. So können dann notwendige Materialien (z.B.:
Nahrungsmittel, Rohstoffe) eingekauft werden. Am Ende des Projekts sollte so viel Geld in
G-Mark erwirtschaftet worden sein, dass der Kredit zurückbezahlt werden kann.

Falls ein Betrieb während des Projekts noch Dinge einkaufen muss, kann er wieder nach Genehmigung
durch das Wirtschaftsministerium einkaufen und sich dann gegen Vorlage des Belegs den
Preis in Euro durch Bezahlung in G-Mark erstatten lassen.

\textit{Beispiel: Der Bäcker braucht für seine Butterbrötchen noch neue Butter, also lässt er sich
den Kauf von 5 Packungen Butter genehmigen, geht dann einkaufen für 5€, bekommt diese gegen Vorlage
des Belegs vom Wirtschaftsministerium zurück und muss dann den Betrag in G-Mark begleichen.}

\section{Steuer}

Zur Finanzierung der Beamtenlöhne und anderer Projektkosten (Druckkosten, Geschenke etc.) ist eine
\textbf{Umsatzsteuer von 25\%} vorgesehen.

Diese wird von den Betrieben bei Verkauf an den Endkunden abgeführt, das heißt beim Handel von
Händler zu Händler wird keine Steuer abgeführt.
Dieser Umstand vereinfacht jegliche Berechnungen im Hinblick auf Vorsteuerabzüge, verändert das
System in der Sache jedoch nicht, da weiterhin die Steuerlast beim Endkunden liegt.

\textit{Beispiel: Ein Bäcker verkauft Butterbrötchen für 1€. Beim Verkauf verlangt er 1,25€ von
denen er 25\%, also 0,25€ an den Staat als Steuer abführt.}

\section{Währung}

Goethopia hat eine eigene Währung, sie trägt den Namen \textbf{"`G-Mark"'}. Der Wechselkurs
beträgt \textbf{1:10}, d.h. ein Euro sind 10 G-Mark.

Das hat folgenden Hintergrund: Da wir kein Kleingeld drucken wollen, sondern die kleinste Einheit
1 G-Mark ist, wir aber gleichzeitig eine genauere Differenzierung der Preise benötigen (z.B. um
die 25\% Umsatzsteuer zu realisieren), brauchen wir als kleinste Einheit 10 Cent und damit 1 G-Mark.
Daraus resultiert ein Wechselkurs von 1:10.

Bei einem geplanten Umsatz von ca. 14.000€ (siehe Haushalt) muss entsprechend viel Währung gedruckt
werden. Wir sind jetzt einfach mal von 20.000€ also 200.000 G-Mark ausgegangen.
Die Stückelung sieht dabei folgendermaßen aus:

\vspace{5mm}
\begin{tabular}{R{2cm}R{4cm}}
    \toprule
    Anzahl & Wert in G-Mark \\
    \midrule
    2.000 & 1 \\
    2.000 & 2 \\
    3.000 & 5 \\
    3.000 & 10 \\
    2.000 & 20 \\
    1.000 & 50 \\
    200 & 100 \\
    200 & 200 \\
    \bottomrule
\end{tabular}

Der Schwerpunkt liegt also auf den kleinen Scheinen, besonders aber auf der Mindestlohn Kombination
10 G-Mark + 5 G-Mark.

Insgesamt müssen also \textbf{13.400 Scheine} gedruckt werden. Die \textbf{Fälschungssicherheit}
ist noch nicht
gelöst.

\section{Entlohnung}

Jeder Bürger muss mindestens den \textbf{Mindestlohn von 1,50€} die Stunde erhalten, bei
\textbf{3 Stunden} Arbeitszeit ergibt das ein Mindesteinkommen von \textbf{4,50€} pro Tag.

Begründung: Jeder Arbeiter sollte sich von einer Stunde Arbeit wenigstens drei Brötchen kaufen
können, also 3 $\cdot$ 0,50€ = 1,50€, da dann wenigstens etwas erzielt wird mit der Arbeit.

\section{Zoll und Visa}

Für externe Besucher des Staates Goethopia, muss ein Visum gekauft werden. Dieses kostest für
Erwachsene 10€, wobei allerdings 8€ gegen G-Mark getauscht und 2€ als Gebühr einbehalten werden.
Kinder unter 14 Jahren zahlen 6€ (5€ gegen G-Mark, 1€ Gebühr) und Familien müssen nur die ersten
zwei Kinder bezahlen.

Vor jedem Eingang können sowohl Euros in G-Mark als auch Visa gekauft werden, außerdem werden die
Ausweise bzw. die Anwesenheiten kontrolliert.
Dazu sind zwei Zollbeamte und zwei Kontrolleure am Haupteingang und ein Kontrolleur und ein
Zollbeamter am Eingang zur Innenstadtsporthalle vorgesehen (ingesamt 6 Zollbeamte).

\newpage

\section{Haushalt}

\subsection{Beamtengehälter}

Den größten Anteil in unserem Haushalt machen die Beamtengehälter, dabei werden (fast) alle
Beamten für den ganzen Tag bezahlt.

Für bedeutende Funktionäre ist ein höherer Lohn vorgesehen (insbesondere Verfassungsrichter, Kanzlerin und Präsidentin). Alle Beamten erhalten mehr als den Mindestlohn.

Für das Erdgeschoss und das erste bis dritte Obergeschoss sind insgesamt vier, für das Untergeschoss
mit Eingangsbereich weitere zwei, Polizisten vorgesehen.

\vspace{5mm}

\begin{lohnrechnung}
    \posten{Verfassungsrichter} {3}  {7} {84}
    \posten{Richter + Schöffen} {4}  {7} {112}
    \posten{Staatsanwalt}       {2}  {7} {56}
    \posten{Minister}           {5}  {7} {140}
    \posten{Kanzlerin}          {1}  {7} {28}
    \posten{Präsidentin}        {1}  {6} {24}
    \posten{Abgeordnete}        {30} {7} {840}
    \posten{Staatssekretäre}    {15} {6} {360}
    \posten{Polizeichef}        {1}  {6} {24}
    \posten{Polizei + Zoll}     {24} {6} {576}
    \posten{WikD}               {8}  {6} {192}
\end{lohnrechnung}

\newcounter{beamten}
\addtocounter{beamten}{\thetotal}

\vspace{5mm}

Zusätzlich zu obiger Tabelle gilt für die Entlohnung der Parlamentarier, dass für jede
unentschuldigte Sitzung ab dem 1.6.18 der Lohn pro Tag, um 1€ sinkt.

\subsection{Ausgaben}

\begin{kostenrechnung}
    \posten{Druckkosten}{Schätzung}{keine Ahnung}{130}
    \posten{Webseite}{5€ im Monat, 10 Monate lang}{mit genug Zeitpuffer}{50}
    \posten{Geschenke}{Schätzung}{für Staatsbesuche}{50}
    \posten{Kredite}{50 Betriebe à 50€}{Vorschuss, wird im besten Fall zurückgezahlt}{2500}
    \posten{Schüler}{10€ $\cdot$ 700 Bürger}{Rückzahlung der zu Beginn bezahlten 10€}{7000}
    \posten{Beamten}{siehe oben}{}{\thebeamten}
\end{kostenrechnung}

\newpage

\subsection{Einnahmen}

\noindent Im Folgenden wird von einem Umsatz von 12000€ ausgegangen: Insgesamt werden \thebeamten € Lohn
an Beamten und weitere ca. 700 Schüler $\cdot$ 1,50€ Mindestlohn $\cdot$ 3 Stunden
$\cdot$ 4 Tage Lohn von den Betrieben ausgezahlt.
Damit entsteht, angenommen alles erarbeitete Geld wird wieder ausgegeben, ein Gesamtumsatz von
\textbf{13660€}. Hinzu kommen Einnahmen durch Eltern und Besucher.

\noindent Mit Puffer abgeschätzt landen wir also bei mindestens 12000€ Umsatz.

\vspace{5mm}
\begin{kostenrechnung}
    \posten{Sponsoren}{Schätzung}{bereits zugesagt}{3000}
    \posten{Steuern}{25\% $\cdot$ 12000}{siehe oben}{3000}
    \posten{Schüler}{10€ $\cdot$ 700 Bürger}{4€ werden einbehalten, der Rest in G-Mark ausgezahlt}{7000}
    \posten{Kredite}{50 Betriebe à 50€}{im besten Falle zurückbezahlter Kredit}{2500}
\end{kostenrechnung}

\end{document}
